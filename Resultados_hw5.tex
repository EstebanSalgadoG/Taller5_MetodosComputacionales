\documentclass{article}
\usepackage{graphicx}

\begin{document}

\title{Computational Methods}
\author{Esteban Salgado
201225345}

\maketitle

\section{Results}
\subsection{Ionic channels}
The ion channel forming peptide gramicidin A adopts a number of distinct conformations in different environments. The channels specifically conduct monovalent cations (alkali metals, $T1^+, Ag^+, NH^+_4$ and $H^+$). Reports of the atomic structures of gramicidin conformers have included values for the pore/channel radius or diameter. These generally have been based on interatomic distances across the channel, sometimes with the van der Waals radius of the atoms subtracted. 
\\According to the dataset provided in class, we were able to determine the higher size of the ionic channel without making contact with any other particles.
The results are shown in the following diagram.

\begin{figure}[h!]
   \centering
    \includegraphics[width=3.0in]{ej1.jpg}
    \caption{Results using the $Canal_ionico.txt$ dataset}
    \label{simulationfigure}
\end{figure}

\begin{figure}
   \centering
    \includegraphics[width=3.0in]{hist1.jpg}
    \caption{Histogram of the $x,y$ positions}
    \label{histogram}
\end{figure}

\subsection{RC Circuit}
An RC circuit is composed of a resistor R, and a capacitance capacitor C connected in series. If the capacitor is initially discharged and the circuit is closed, the capacitor starts to accumulate charge and The equation describing the charge as a function of time is:

\begin{figure}[h!]
   \centering
$q(t)=Q_{max}\cdot(1-e^{(-t/RC)})$
\end{figure}

Where $Q_{max}$ is the maximum charge reached by the capacitor, C is the capacitance of the capacitor and R is the resistance of the circuit. In an RC circuit there is also a relation between the capacitance C, the voltage V and the maximum charge $Q_{max}$ of the capacitor: $Q_{max} = V_0 \cdot C$. For this exercise, we assume that the potential is $V_0 = 10V$.

\begin{figure}[h!]
   \centering
    \includegraphics[width=3.0in]{RvsLikelihood.jpg}
    \caption{Histogram R values vs Likelihood.}
    \label{histogram2}
\end{figure}

\begin{figure}[h!]
   \centering
    \includegraphics[width=3.0in]{CvsLikelihood.jpg}
    \caption{Histogram C values vs Likelihood.}
    \label{histogram3}
\end{figure}

\begin{figure}[h!]
   \centering
    \includegraphics[width=3.0in]{model.jpg}
    \caption{Model for the best values of R,C.}
    \label{model}
\end{figure}

\end{document}